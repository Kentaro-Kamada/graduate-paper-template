
% パッケージ
\usepackage{booktabs}
\usepackage{longtable}
\usepackage{array}
\usepackage{multirow}
\usepackage{wrapfig}
\usepackage{float}
\usepackage{colortbl}
\usepackage{pdflscape}
\usepackage{tabu}
\usepackage{threeparttable}
\usepackage{threeparttablex}
\usepackage[normalem]{ulem}
\usepackage{makecell}
\usepackage{xcolor}
\usepackage{scalefnt} % 表の大きさいじる

% フォント指定(MS明朝 + Times New Roman)
\usepackage[hiragino-pro]{luatexja-preset}
% \usepackage{luatexja-fontspec}
\setmainfont{Times-New-Roman}
% \setmainjfont[
%     BoldFont           = MS-Gothic,
%     BoldFeatures       = {FakeBold=2},
%     ItalicFont         = MS-Mincho,
%     ItalicFeatures     = {FakeSlant=0.33},
%     BoldItalicFont     = MS-Gothic,
%     BoldItalicFeatures = {FakeBold=2, FakeSlant=0.33}
% ]{MS-Mincho}


% レイアウト設定
\usepackage{geometry}
\geometry{top=3cm, bottom=3cm, left=3cm, right=3cm}

% 文字数マクロ
\makeatletter
\def\mojiparline#1{
    \newcounter{mpl}
    \setcounter{mpl}{#1}
    \@tempdima=\linewidth
    \advance\@tempdima by-\value{mpl}\zw
    \addtocounter{mpl}{-1}
    \divide\@tempdima by \value{mpl}
    \newskip\@tempkskip
    \@tempkskip=\ltjgetparameter{kanjiskip}
    \advance\@tempkskip by \@tempdima
    \ltjsetparameter{kanjiskip=\@tempkskip}
    \advance\parindent by\@tempdima
    }
\makeatother

\def\linesparpage#1{
    \baselineskip=\textheight
    \divide\baselineskip by #1
    }

% 引用文献インデントマクロ定義
\newenvironment{hangall}[1]{\hangindent = #1\zw\everypar{\hangindent = #1\zw}}{}

% 章・節の定義変更
% \renewcommand{\section}{\arabic{section}}
% \renewcommand{\subsection}{\arabic{section}.\arabic{subsection}}


%%%%%% 文末注の設定
% 脚注を文末注に変換
\usepackage{endnotes}
\usepackage{etoolbox}
\let\footnote=\endnote
\renewcommand{\theendnote}{\arabic{endnote})} % 番号を「1)」の形式に
\renewcommand{\notesname}{注} % デフォルトで入る'Note'の文字を'[注]'に変える

% インデント設定
\makeatletter % 「@」を命令中で使用する場合には\makeatletterと\makeatotherではさむ
\renewcommand{\enoteformat}{
  \rightskip\z@ % 右側インデント0文字
  \leftskip=2\zw % 左側インデント全角2文字
  \parindent=1\zw % 先頭インデント全角1文字
  \leavevmode\llap{\hbox{$\theenmark $}}} % 文末注の前の「1)」を上付きじゃなくする(上付きにするときは\theenmarkを^{\theenmark}とする)
\makeatother

\def\enotesize{\fontsize{10.5bp}{19.15bp}\selectfont} % フォント設定


% stata出力の表中のマクロ
\def\sym#1{\ifmmode^{#1}\else\(^{#1}\)\fi}

