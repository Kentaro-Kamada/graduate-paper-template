
% パッケージ
\usepackage{booktabs}
\usepackage{longtable}
\usepackage{array}
\usepackage{multirow}
\usepackage{wrapfig}
\usepackage{float}
\usepackage{colortbl}
\usepackage{pdflscape}
\usepackage{tabu}
\usepackage{threeparttable}
\usepackage{threeparttablex}
\usepackage[normalem]{ulem}
\usepackage{makecell}
\usepackage{xcolor}
\usepackage{scalefnt} % 表の大きさいじる

% フォント指定(MS明朝 + Times New Roman)
\usepackage{luatexja-fontspec}
\setmainfont{Times New Roman}
\setmainjfont{MS Mincho}

% レイアウト設定
\usepackage{geometry}
\geometry{top=3cm, bottom=3cm, left=3cm, right=3cm}

% 文字数マクロ
\makeatletter
\def\mojiparline#1{
    \newcounter{mpl}
    \setcounter{mpl}{#1}
    \@tempdima=\linewidth
    \advance\@tempdima by-\value{mpl}\zw
    \addtocounter{mpl}{-1}
    \divide\@tempdima by \value{mpl}
    \newskip\@tempkskip
    \@tempkskip=\ltjgetparameter{kanjiskip}
    \advance\@tempkskip by \@tempdima
    \ltjsetparameter{kanjiskip=\@tempkskip}
    \advance\parindent by\@tempdima
    }
\makeatother

\def\linesparpage#1{
    \baselineskip=\textheight
    \divide\baselineskip by #1
    }

% 引用文献インデントマクロ定義
\newenvironment{hangall}[1]{\hangindent = #1\zw\everypar{\hangindent = #1\zw}}{}


% 脚注を文末注に変換
\usepackage{endnotes}
\let\footnote=\endnote
\renewcommand{\theendnote}{\arabic{endnote})} % 番号を「1)」の形式に
\renewcommand{\notesname}{[注]} % デフォルトで入る'Note'の文字を'[注]'に変える
\usepackage{etoolbox}
\patchcmd{\enoteformat}{1.8em}{1\zw}{}{} % インデント設定
\def\enotesize{\fontsize{10.5pt}{21.9pt}\selectfont} % フォント設定